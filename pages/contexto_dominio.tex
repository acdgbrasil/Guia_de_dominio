\section{Entendendo o Domínio do Projeto}

% Esta seção define o problema central que o sistema resolve.
% O objetivo aqui é garantir que todo o projeto esteja alinhado com as reais necessidades do negócio.
% Para escrever esta seção, responda às perguntas do "Guia Rápido de DDD", na seção correspondente.

\subsection{Descrição Geral}
% Aqui, você deve fornecer uma visão geral do problema e do contexto do negócio.
% Imagine que você está explicando para alguém que nunca ouviu falar do seu projeto.

% Perguntas para te ajudar:
% - Qual é o problema central que o seu sistema resolve?
% - Quem são os principais usuários do sistema e quais problemas enfrentam?
% - Como este sistema se encaixa no fluxo de trabalho do negócio?
% - Existem soluções já utilizadas atualmente? Se sim, quais as limitações delas?

Descreva de maneira objetiva o que o sistema faz e qual problema ele resolve.  
Explique os desafios que os usuários enfrentam hoje e como o sistema melhora esse cenário.  
Caso existam soluções alternativas, detalhe por que elas não são ideais e o que o sistema propõe de diferente.  

\subsection{Público-Alvo e Usuários}
% Esta parte foca nas pessoas que utilizarão o sistema. Defina claramente os usuários e seus objetivos.

% Perguntas para te ajudar:
% - Quem são os principais usuários do sistema e quais problemas enfrentam?
% - O sistema será utilizado apenas internamente ou por clientes externos?
% - Existem diferentes tipos de usuários? Se sim, quais suas diferenças?
% - Algum grupo de usuários tem necessidades especiais (ex: acessibilidade, segurança)?
% - Como os usuários interagem com o sistema no dia a dia?

Liste os perfis de usuários e explique como cada grupo interage com o sistema.  
Caso existam diferentes níveis de acesso ou funcionalidades exclusivas, detalhe essas diferenças.  
Se houver preocupações com acessibilidade ou requisitos especiais, mencione aqui.  

\subsection{Impacto e Benefícios do Sistema}
% Agora que já explicamos o problema e os usuários, é hora de mostrar o impacto positivo do sistema.

% Perguntas para te ajudar:
% - Quais são as operações mais críticas do sistema e por quê?
% - Quais benefícios o sistema trará para a empresa e seus usuários?
% - Quais métricas podem ser usadas para medir o sucesso do sistema?
% - Se este sistema deixasse de existir amanhã, quais seriam as maiores dificuldades para a empresa?

Destaque os principais **benefícios** do sistema para os usuários e para a empresa.  
Explique quais **processos são melhorados ou automatizados** e como isso gera **eficiência**.  
Caso seja possível, apresente métricas ou formas de medir o impacto positivo do sistema.  

\subsection{Fluxo de Trabalho e Integração com Outros Sistemas}
% Descreva como o sistema se encaixa no dia a dia da empresa.

% Perguntas para te ajudar:
% - Como este sistema se encaixa no fluxo de trabalho do negócio?
% - O sistema depende de outras ferramentas para funcionar?
% - Existem integrações com outros sistemas? Se sim, quais?
% - O sistema pode operar de forma independente ou precisa de outras tecnologias?

Descreva o **fluxo operacional** do sistema dentro da empresa.  
Explique se ele funciona de maneira independente ou se depende de outras ferramentas e serviços.  
Caso existam **integrações com sistemas externos** (ex: APIs, ERPs, CRMs), detalhe quais são e o motivo dessas conexões.  

\subsection{Regras e Restrições do Domínio}
% Defina os limites do sistema e suas regras principais.

% Perguntas para te ajudar:
% - Existem leis ou regulamentações que o sistema precisa seguir?
% - Existem restrições técnicas ou operacionais no domínio?
% - O sistema precisa lidar com diferentes moedas, idiomas ou fuso-horários?
% - Há alguma necessidade específica de segurança e privacidade?

Liste as **restrições** que impactam diretamente o funcionamento do sistema, como:  
\begin{itemize}
    \item Regulamentações (ex: LGPD, GDPR).
    \item Restrições técnicas (ex: só pode operar online, compatibilidade com navegadores específicos).
    \item Necessidades de segurança (ex: criptografia de dados, controle de acesso).
\end{itemize}

\subsection{Resumindo o Domínio}
% Aqui você criará um resumo final do domínio do projeto.
% Esse resumo será útil para qualquer pessoa que precise entender rapidamente o propósito do sistema.
% Para isso, siga as instruções abaixo.

% Como estruturar o resumo:
% 1️⃣ Escreva UMA frase curta explicando o propósito central do sistema.
% 2️⃣ Explique, em no máximo dois parágrafos, os principais problemas que ele resolve.
% 3️⃣ Liste, em bullet points, os usuários principais e seus objetivos no sistema.
% 4️⃣ Resuma quais são as integrações do sistema com outros serviços e ferramentas.
% 5️⃣ Indique as principais regras e restrições que o sistema deve seguir.

O resumo final deve sintetizar todas as informações documentadas acima, garantindo que qualquer pessoa possa entender rapidamente o domínio do sistema.  

\textbf{Sugestão de estrutura para o resumo:}
\begin{itemize}
    \item O \textbf{[Nome do Sistema]} é uma plataforma desenvolvida para \textbf{[propósito principal]}.
    \item Ele foi criado para resolver \textbf{[problema central]}, garantindo que \textbf{[benefício esperado]}.
\end{itemize}

\textbf{Problemas Resolvidos:}
\begin{itemize}
    \item O sistema permite que \textbf{[usuários principais]} consigam \textbf{[ação específica]}.
    \item Automatiza \textbf{[processo]}, reduzindo \textbf{[problema enfrentado]}.
    \item Substitui \textbf{[solução antiga]} por um modelo mais eficiente, trazendo \textbf{[benefício]}.
\end{itemize}

\textbf{Integrações e Dependências:}
\begin{itemize}
    \item Se conecta com \textbf{[sistema externo]} para \textbf{[finalidade da integração]}.
    \item Opera de forma \textbf{[independente/necessita de X ferramenta]}.
    \item Possui suporte para \textbf{[idiomas, moedas, diferentes plataformas]}.
\end{itemize}

\textbf{Principais Regras do Domínio:}
\begin{itemize}
    \item O sistema segue as diretrizes de \textbf{[norma ou regulamentação]}.
    \item Somente usuários \textbf{[perfil específico]} podem acessar determinadas funções.
    \item Deve operar com \textbf{[restrição técnica]}, garantindo \textbf{[segurança/performance]}.
\end{itemize}

% Dica Final:
% O resumo deve ser CLARO, OBJETIVO e direto ao ponto. Evite detalhes técnicos complexos.
% A ideia é que qualquer pessoa, ao ler esta parte, entenda rapidamente o que o sistema faz e como ele funciona.
