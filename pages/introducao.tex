\section{Introdução}

% Diretriz para escrita da introdução:
% A introdução deve apresentar um resumo geral do documento, contextualizando o tema principal.
% Deve responder às seguintes perguntas:
% - O que é Domain-Driven Design (DDD)?
% - Por que ele é relevante para este projeto?
% - Quais são os principais benefícios da abordagem DDD?
% - Qual o objetivo deste documento?
% Além disso, a introdução deve fornecer uma visão geral das seções que serão abordadas.

\textbf{Sugestão de Estrutura:}

\begin{itemize}
    \item \textbf{Parágrafo 1:} Explicar o conceito de Domain-Driven Design (DDD), sua origem e seu papel no desenvolvimento de software.
    \item \textbf{Parágrafo 2:} Apresentar a importância do DDD no contexto do projeto e como ele impacta a modelagem do domínio.
    \item \textbf{Parágrafo 3:} Destacar os principais benefícios da abordagem DDD, como alinhamento entre equipe técnica e negócio, manutenibilidade e escalabilidade.
    \item \textbf{Parágrafo 4:} Explicar o objetivo deste documento e como ele servirá como um guia para a equipe.
    \item \textbf{Parágrafo 5:} Resumir as seções que serão abordadas no documento.
\end{itemize}

% Após preencher os tópicos acima, remover estas diretrizes e redigir o conteúdo definitivo.
