\section{Guia Rápido de DDD}

% Este guia serve como referência prática para modelar um sistema utilizando Domain-Driven Design (DDD).
% As perguntas abaixo ajudam a estruturar um modelo de DDD, mesmo sem experiência prévia.
% NÃO É NECESSÁRIO RESPONDER TODAS AS PERGUNTAS! Cada empresa pode adaptar o documento às suas necessidades.

% REFERÊNCIAS PRINCIPAIS PARA APROFUNDAMENTO:
%  Implementando Domain-Driven Design - Vaughn Vernon
%  Domain-Driven Design: Tackling Complexity in the Heart of Software - Eric Evans

% ╰(*°▽°*)╯ SESSÕES ESSENCIAIS PARA UM DDD COMPLETO: ╰(*°▽°*)╯
% - Entendendo o Domínio do Projeto
% - Definição de Contextos e Fronteiras
% - Modelagem do Domínio
% - Estrutura de Objetos e Regras de Negócio
% - Integração entre Sistemas e Comunicação entre Contextos

% As demais seções são opcionais e dependem da complexidade do sistema.

\subsection{Entendendo o Domínio do Projeto}
% Esta seção ajuda a definir o problema central que o sistema resolve.
% As perguntas garantem que o modelo de domínio tenha um propósito claro.

\begin{itemize}
    \item Qual é o problema central que o seu sistema resolve?
    \item Quem são os principais usuários do sistema e quais problemas enfrentam?
    \item Como este sistema se encaixa no fluxo de trabalho do negócio?
    \item Existem soluções já utilizadas atualmente? Se sim, quais as limitações delas?
    \item Quais são as operações mais críticas do sistema e por quê?
    \item Se este sistema deixasse de existir amanhã, quais seriam as maiores dificuldades para a empresa?
\end{itemize}

\subsection{Definição de Contextos e Fronteiras}
% Define os Contextos Delimitados do sistema. Essencial para evitar confusão conceitual entre diferentes partes do projeto.

\begin{itemize}
    \item Quais partes do sistema podem ser separadas como contextos independentes?
    \item Como garantir que diferentes áreas do sistema não criem conflitos entre si?
    \item O que define os limites entre os Contextos Delimitados do seu projeto?
    \item Alguma equipe ou área de negócio específica é responsável por cada contexto?
    \item Como esses contextos interagem entre si e com sistemas externos?
    \item Algum contexto precisa compartilhar dados ou lógica com outro contexto?
\end{itemize}

\subsection{Subdomínios e Estratégia de Negócio}
% Identifica o Core Domain e os subdomínios auxiliares. Essencial para saber o que deve ser priorizado no sistema.

\begin{itemize}
    \item Qual é o **Core Domain** (Domínio Principal) do seu sistema?
    \item Existem funcionalidades secundárias que não fazem parte do Core Domain?
    \item O sistema depende de algum serviço terceirizado ou integração externa?
    \item O seu sistema pode ser dividido em múltiplos subdomínios? Se sim, quais?
    \item Como os subdomínios se relacionam com o Core Domain?
    \item Quais subdomínios precisam ser implementados primeiro para o sistema funcionar?
\end{itemize}

\subsection{Modelagem do Domínio}
% Aqui definimos as Entidades do sistema e como elas interagem.
% ESSENCIAL para um DDD funcional.

\begin{itemize}
    \item Quais são as entidades centrais do seu domínio?
    \item Como essas entidades interagem entre si dentro do contexto do sistema?
    \item Quais atributos e comportamentos são essenciais para cada entidade?
    \item Existem regras de negócio que precisam ser validadas dentro das entidades?
    \item Alguma entidade precisa garantir consistência de dados com outra?
    \item Existem operações críticas que precisam de transações para garantir integridade?
\end{itemize}

\subsection{Estrutura de Objetos e Regras de Negócio}
% Explica como estruturar os Agregados, Objetos de Valor e Serviços de Domínio.

\begin{itemize}
    \item Quais atributos ou comportamentos podem ser representados por Objetos de Valor?
    \item Existem cálculos ou transformações que se repetem e podem ser encapsulados?
    \item Algum conjunto de entidades deve ser tratado como uma única unidade lógica?
    \item O que deve ser considerado na definição dos Agregados?
    \item Como garantir que as regras de negócio sejam centralizadas e não espalhadas no código?
    \item Alguma regra de negócio precisa ser compartilhada entre diferentes partes do sistema?
\end{itemize}

\subsection{Integração entre Sistemas e Comunicação entre Contextos}
% ESSENCIAL para sistemas que precisam interagir com serviços externos.

\begin{itemize}
    \item Como o sistema se comunica com serviços externos ou sistemas legados?
    \item Alguma parte do sistema precisa se proteger contra mudanças em sistemas externos?
    \item Existe alguma regra de negócio que não pode ser comprometida por integrações externas?
    \item Como garantir que a comunicação entre contextos não cause acoplamento excessivo?
    \item Quando deve ser utilizada mensageria em vez de chamadas diretas entre contextos?
    \item Como lidar com falhas na comunicação entre sistemas?
\end{itemize}

\subsection{Implementando DDD em Projetos Existentes}
% Seção útil para empresas que querem migrar um sistema legado para DDD.

\begin{itemize}
    \item Quais são as maiores dificuldades do sistema atual que o DDD pode resolver?
    \item Alguma parte do sistema pode ser separada como um novo contexto sem impactar o resto?
    \item Como dividir a migração do sistema para DDD em pequenas etapas?
    \item Existe alguma funcionalidade que pode ser usada como um piloto para testar a abordagem?
    \item Como garantir que a transição para DDD não cause interrupções na operação do sistema?
    \item O que fazer se o modelo de domínio precisar ser reformulado no meio do processo?
\end{itemize}

\subsection{Melhores Práticas para um Modelo Sustentável}
% Estratégias para manter o modelo de domínio atualizado e funcional no longo prazo.

\begin{itemize}
    \item Como garantir que o modelo de domínio continue evoluindo sem perder consistência?
    \item Como manter a comunicação entre equipe técnica e especialistas do negócio?
    \item Quais são os sinais de que um modelo de domínio precisa ser ajustado?
    \item Como evitar que a modelagem do domínio se torne desatualizada?
    \item Quais estratégias ajudam a manter o código do domínio limpo e organizado?
    \item Como documentar um modelo DDD de maneira eficiente sem burocracia excessiva?
\end{itemize}

% Esta seção serve como um guia prático de DDD. Não é necessário responder todas as perguntas, apenas as mais relevantes para o sistema.
% Para mais informações, consulte:
% Implementando Domain-Driven Design - Vaughn Vernon
% Domain-Driven Design: Tackling Complexity in the Heart of Software - Eric Evans
